%!TEX root = ../../main.tex


\begin{figure}[!htb]

\makeatletter
\newcommand*\derivesubmol[4]{%
    \saveexpandmode\saveexploremode\expandarg\exploregroups
    \csname @\ifcat\relax\noexpand#2first\else second\fi oftwo\endcsname
        {\expandafter\StrSubstitute\@car#2\@nil}
        {\expandafter\StrSubstitute\csname CF@@#2\endcsname}
    {\@empty#3}{\@empty#4}[\temp@]%
    \csname @\ifcat\relax\noexpand#1first\else second\fi oftwo\endcsname
        {\expandafter\let\@car#1\@nil}
        {\expandafter\let\csname CF@@#1\endcsname}\temp@
    \restoreexpandmode\restoreexploremode
}
\makeatother

\renewcommand*\printatom[1]{\ensuremath{\mathsf{#1}}}

\setatomsep{2.5em}
\setcrambond{1ex}{1pt}{2pt}

\definesubmol{rt1}{-[2]rt1}
\definesubmol{rt2}{-[6,.6]rt2}

\definesubmol{ribose}{%
	-[::-90,1.5]%
	    (%
	        -[::115,1.35]O%
	        -[::-50,1.35]%
	    )%
	<[::45,1]%
	    (%
	      -[::45,,,,line width=3pt,shorten <=-.5pt,shorten >=-.5pt]%
	          (!{rt2})%
	      >[::45,1]%
	          (!{rt1})%
	     )%
}

\definesubmol{OH_group}{-[::-45,1.2]OH}

\definesubmol{phosphate}{%
	-[::-45,1.2]O%
	-[::0,1.5]P%
	(%
		-[::-90,1.5]O{^-}%
	)%
	(%
		=[::0,1]O%
	)%
	-[::90,1.5]O%
}

\definesubmol{phosphate_lhs_beg}{%
	O{^-}%
	-[::-90,1.5]P%
	(%
		=[::0,1]O%
	)%
	(%
		-[::-90,1.5]O{^-}%
	)%
	-[::90,1.5]O%
}

\definesubmol{phosphate_rhs_beg}{%
	O{^-}%
	-[::90,1.5]P%
	(%
		-[::-90,1.5]O{^-}%
	)%
	(%
		=[::0,1]O%
	)%
	-[::90,1.5]O%
}

\definesubmol{adenine}{%
	N*5%
	(%
		[:72]-*6%
		(%
			-N=-N%
			(%
				-[:15,2.3,,,dash pattern=on 1pt off 1pt]%
			)%
			=(%
			-NH_2%
			-[:16,2.2,,,dash pattern=on 1pt off 1pt]%
			)-%
		)=-N=-%
	)%
}
\definesubmol{adenineR}{%
	N*5%
	([:138]-=N-*6%
	(-(-[,,,2]H_2N%
	)=N(%
	)-=N-)=-)%
}

\definesubmol{guanine}{%
	N*5%
	(%
		[:72]-*6%
		(%
			-N=%
			(%
				-NH_2%
				(-[:16,2.2,,,dash pattern=on 1pt off 1pt])%
			)%
			-NH%
			(-[:15,2.4,,,dash pattern=on 1pt off 1pt])-%
			(%
			=O(-[:16,2.1,,,dash pattern=on 1pt off 1pt])%
			)-%
		)%
		=-N=-%
	)%
}

\definesubmol{guanineR}{%
	N*5%
	(%
		[:138]-=N-*6%
		(-%
			(%
				=O%
			)%
				-[,,,2]HN-%
			(%
				-[,,,2]H_2N%
			)%
			=N-%
		)%
		=-%
	)%
}

\definesubmol{cytosine}{%
	N*6%
	([:60]-%
		(%
			=O%
			(-[:22,2.05,,,dash pattern=on 1pt off 1pt])%
		)%
		-N%
		(-[:20,2.3,,,dash pattern=on 1pt off 1pt])%
		=%
		(%
			-NH_2%
			(-[:22,2.2,,,dash pattern=on 1pt off 1pt])%
		)%
		-=-%
	)%
}
\definesubmol{cytosineR}{%
	N*6([:150]-=-(-[,,,2]H_2N%
	)=N-%
	(=O)-)%
}

\definesubmol{thymine}{%
	N*6%
	([:60]-%
		(%
			=O%
		)%
		-NH%
		(%
			(-[:22,2.3,,,dash pattern=on 1pt off 1pt])%
		)-%
		(%
			=O%
			(-[:22,2.1,,,dash pattern=on 1pt off 1pt])%
		)-%
		(-)%
		=-%
	)%
}
\definesubmol{thymineR}{%
	N*6([:150]-=(-)-%
	(=O)%
	-[,,,2]HN%
	-(=O)-)%
}

\derivesubmol{deoxyribose}{ribose}{(!{rt2})}{}

\derivesubmol{A_group_lhs}{deoxyribose}{!{rt1}}{-[::-45,1.5]!{adenine}}
\derivesubmol{G_group_lhs}{deoxyribose}{!{rt1}}{-[::-45,1.5]!{guanine}}
\derivesubmol{C_group_lhs}{deoxyribose}{!{rt1}}{-[::-45,1.5]!{cytosine}}
\derivesubmol{T_troup_lhs}{deoxyribose}{!{rt1}}{-[::-45,1.5]!{thymine}}

\derivesubmol{A_group_rhs}{deoxyribose}{!{rt1}}{-[::-45,1.5]!{adenineR}}
\derivesubmol{G_group_rhs}{deoxyribose}{!{rt1}}{-[::-45,1.5]!{guanineR}}
\derivesubmol{C_group_rhs}{deoxyribose}{!{rt1}}{-[::-45,1.5]!{cytosineR}}
\derivesubmol{T_group_rhs}{deoxyribose}{!{rt1}}{-[::-45,1.5]!{thymineR}}


\centering
%\hrulefill \\
%\vspace{5pt}


\tikzset{
    ncbar angle/.initial=90,
    ncbar/.style={
        to path=(\tikztostart)
        -- ($(\tikztostart)!#1!\pgfkeysvalueof{/tikz/ncbar angle}:(\tikztotarget)$)
        -- ($(\tikztotarget)!($(\tikztostart)!#1!\pgfkeysvalueof{/tikz/ncbar angle}:(\tikztotarget)$)!\pgfkeysvalueof{/tikz/ncbar angle}:(\tikztostart)$)
        -- (\tikztotarget)
    },
    ncbar/.default=0.25cm,
}
\tikzset{Lbrace/.style={ncbar=-0.25cm}}
\tikzset{Rbrace/.style={ncbar=0.25cm}}

\scalebox{0.9}{%
\begin{tikzpicture}[txt/.style={font=\sffamily\slshape\bfseries\scriptsize,text=gray!60,text width=2.5cm}]
\tiny

% \filldraw[even odd rule,draw=white,inner color=red!40,outer color=white] (0.5,6.95) circle (0.5);
% \filldraw[even odd rule,draw=white,inner color=red!40,outer color=white] (-0.35,6.95) circle (0.4);
% \filldraw[even odd rule,draw=white,inner color=green!50,outer color=white] (0.95,4.4) circle (0.5);
% \filldraw[even odd rule,draw=white,inner color=yellow!80,outer color=white] (2.15,1.8) circle (0.5);
% \filldraw[even odd rule,draw=white,inner color=blue!25,outer color=white] (4,-0.78) circle (0.5);
% \filldraw[even odd rule,draw=white,inner color=blue!25,outer color=white] (3.15,-0.78) circle (0.4);
%
% \filldraw[even odd rule,draw=white,inner color=yellow!80,outer color=white] (2.875,7.8) circle (0.5);
% \filldraw[even odd rule,draw=white,inner color=blue!25,outer color=white] (3.45,4.855) circle (0.5);
% \filldraw[even odd rule,draw=white,inner color=blue!25,outer color=white] (4.2,5.255) circle (0.4);
% \filldraw[even odd rule,draw=white,inner color=red!40,outer color=white] (4.65,2.25) circle (0.5);
% \filldraw[even odd rule,draw=white,inner color=red!40,outer color=white] (5.35,2.65) circle (0.4);
% \filldraw[even odd rule,draw=white,inner color=green!50,outer color=white] (6.35,0.05) circle (0.5);

\node[draw=white] (UL) at (-5,9.5) {};
\node[draw=white] (UR) at (11,9.5) {};
\node[draw=white] (LL) at (-5,-3) {};
\node[draw=white] (LR) at (11,-3) {};

\draw [draw=gray!40,ultra thick] (-0.75,8.25) to [Lbrace] (-0.75,5.85);
\draw [draw=gray!40,ultra thick] (0.4,5.5) to [Lbrace] (0.4,3.1);
\draw [draw=gray!40,ultra thick] (1.5,2.85) to [Lbrace] (1.5,0.55);
\draw [draw=gray!40,ultra thick] (2.75,0.4) to [Lbrace] (2.75,-1.95);

\node[txt] at (0.36,6.05) {Adenine};
\node[txt] at (1.5,3.3) {Cytosine};
\node[txt] at (2.6,0.725) {Thymine};
\node[txt] at (3.85,-1.76) {Guanine};

\draw [draw=gray!40,ultra thick] (3.625,8.9) to [Rbrace] (3.625,6.25);
\draw [draw=gray!40,ultra thick] (4.75,6) to [Rbrace] (4.75,3.8);
\draw [draw=gray!40,ultra thick] (5.9,3.5) to [Rbrace] (5.9,1.275);
\draw [draw=gray!40,ultra thick] (7.1,0.9) to [Rbrace] (7.1,-1.55);

\node[txt] at (3.95,6.4) {Thymine};
\node[txt] at (5.11,3.99) {Guanine};
\node[txt] at (6.27,1.445) {Adenine};
\node[txt] at (7.41,-1.4) {Cytosine};

\node[txt] at (-2.65,8.55) {5' end};
\node[txt] at (6,9) {3' end};
\node[txt] at (2.1,-2.65) {3' end};
\node[txt] at (10.65,-2.2) {5' end};

\node at (1,3) {%
\chemfig{%
  !{phosphate_lhs_beg}%
	!{A_group_lhs}!{phosphate}%
	!{C_group_lhs}!{phosphate}%
	!{T_troup_lhs}!{phosphate}%
	!{G_group_lhs}!{OH_group}%
}};%
\node at (6.25,3.5) {%
\chemfig{%
  !{phosphate_rhs_beg}%
	!{C_group_rhs}!{phosphate}%
	!{A_group_rhs}!{phosphate}%
	!{G_group_rhs}!{phosphate}%
	!{T_group_rhs}!{OH_group}%
}};
\end{tikzpicture}%
}

\Caption{The chemical structure of DNA}
{
The \gls{dna} molecule is a long chain polymer of individual nucleotide building blocks.
Each nucleotide is composed of a phosphate residue, a deoxyribose sugar (pentose), and a nucleobase (adenine, guanine, cytosine, or thymine).
The phospho-deoxyribose subunits are identical at each nucleotide and form the backbone of a \gls{dna} strand along which the sequence of nucleobases may vary.
The \gls{dna} in living cells is typically composed of \n{2} complementary strands, which are connected through hydrogen bonds between complementary nucleobases.
The figure shows a hypothetical sequence of \n{4} base pairs, where hydrogen bonds (\emph{dotted} lines) can only be formed between nucleobases as indicated.}
{fig:info_dna}
% \vspace{-5pt}
% \hrulefill%
\end{figure}
