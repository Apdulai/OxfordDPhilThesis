



%
\subsection{Age estimation in presence of genotype error}
%




%
\section{Investigation of pathogenic rare variants in the UK\,Biobank}
%

large-scale
which consist of hundreds of thousands of individuals

\gls{ukb}
\citep{Sudlow:2015gl}

interim data release

released genotype data on \n{152328} individuals

\footnote{UK\,Biobank: \url{http://www.ukbiobank.ac.uk/} \accessed{2016}{12}{17}}

phased using \tagname{SHAPEIT} version~3 \citep{OConnell:2016dv}



The age estimation method was applied to imputed data, due to the low number of called genotypes available in the \gls{ukb} dataset.
For instance, \n{61969} variants were called on chromosome~1, but which was too low to expect informative results from the implemented IBD detection methods.
Typically, genotyping methods target only a subset of known variant sites in the genome, but which makes it unlikely to find sites which can satisfy the breakpoint conditions in either the \gls{fgt} or \gls{dgt}.
Likewise, sites for which genotype information is available may sit too far apart for the \gls{hmm}-based method, such that transitions from the \emph{ibd} state to the \emph{non} state may occur right away.
This was confirmed in preliminary tests on called genotype data in \gls{ukb}, by randomly selecting \n{100} variants below 1\% allele frequency on chromosome~1, where none of the implemented IBD detection methods was able to infer distinguishable IBD segments.


imputed from reference data by the \gls{hrc}\footnote{Documentation of genotype imputation in the UK\,Biobank (interim data release): \url{http://www.ukbiobank.ac.uk/wp-content/uploads/2014/04/imputation_documentation_May2015.pdf} \accessed{2016}{12}{27}}\footnote{Haplotype Reference Consortium: \url{http://www.haplotype-reference-consortium.org} \accessed{2016}{12}{28}}




%
% !TEX root = ../../main.tex


\begin{table}[!htbp]
\Caption{Number of SNP markers in the UK\,Biobank dataset (interim data release)}
{...

...}
{tab:ukb_size}
\centering
\small
\begin{tabular}{rrrr}
\toprule
Chr. & Called SNPs & Imputed SNPs & Percent imputed \\
\midrule
 1  &  \num{61969}  &  \num{5452964}  &  \dec{98.86357}\%  \\
 2  &  \num{61240}  &  \num{5976762}  &  \dec{98.97536}\%  \\
 3  &  \num{51261}  &  \num{4928206}  &  \dec{98.95984}\%  \\
 4  &  \num{47698}  &  \num{4847493}  &  \dec{99.01603}\%  \\
 5  &  \num{45347}  &  \num{4469408}  &  \dec{98.98539}\%  \\
 6  &  \num{54159}  &  \num{4264536}  &  \dec{98.73001}\%  \\
 7  &  \num{41924}  &  \num{3983600}  &  \dec{98.94759}\%  \\
 8  &  \num{38654}  &  \num{3908598}  &  \dec{99.01105}\%  \\
 9  &  \num{33861}  &  \num{3002182}  &  \dec{98.87212}\%  \\
10  &  \num{37778}  &  \num{3377767}  &  \dec{98.88157}\%  \\
11  &  \num{38730}  &  \num{3426344}  &  \dec{98.86964}\%  \\
12  &  \num{36473}  &  \num{3256167}  &  \dec{98.87988}\%  \\
13  &  \num{26316}  &  \num{2415768}  &  \dec{98.91066}\%  \\
14  &  \num{25379}  &  \num{2234591}  &  \dec{98.86427}\%  \\
15  &  \num{24237}  &  \num{2023911}  &  \dec{98.80247}\%  \\
16  &  \num{28015}  &  \num{2274053}  &  \dec{98.76806}\%  \\
17  &  \num{27088}  &  \num{1969478}  &  \dec{98.62461}\%  \\
18  &  \num{21978}  &  \num{1922435}  &  \dec{98.85676}\%  \\
19  &  \num{24483}  &  \num{1544548}  &  \dec{98.41488}\%  \\
20  &  \num{19730}  &  \num{1540220}  &  \dec{98.71901}\%  \\
21  &  \num{11298}  &   \num{930342}  &  \dec{98.78561}\%  \\
22  &  \num{12622}  &   \num{933978}  &  \dec{98.64858}\%  \\
\midrule
Total & \num{770240} & \num{68683373} & \dec{98.87856}\% \\
\bottomrule
\end{tabular}
\end{table}

%


Here, it was attempted to also phase the imputed dataset using a novel and fast phasing algorithm, \tagname{EAGLE} version~2.6 \citep{loh2016fast,Loh:2016bl}, which was reported to have been tested on genotype call data from \gls{ukb}\footnote{See description in the \tagname{EAGLE2} manual: \url{https://data.broadinstitute.org/alkesgroup/Eagle/} \accessed{2017}{01}{02}}.
The software was reported to achieve higher speeds than other phasing methods while maintaining high accuracy.
This increase in processing speed derives from efficient haplotype matching
using the \gls{pbwt} algorithm \citep{Durbin:2014de}.
Although \tagname{EAGLE2} scales linearly with the number of samples and \glspl{snp}, the attempt to phase the imputed dataset was abandoned, since a test run on chromosome~7 (${\approx 4}$~million \glspl{snp}) did not finish after \n{3} weeks of parallel processing on a high-performance computer with 48 cores and 1TB of memory (using default programme parameters).




Because data contained missing genotypes
\gls{hmm} was modified to skip sites at which the genotype pair was undefined, such that transition probabilities were recalculated

Average rate of pairwise missing genotypes was  \dec{0.007086994}\%~(±\num{5.630e-06}\%~SE)





% %
% \subsection{Uncertainty of inferred IBD breakpoints}
% \label{sec:hardsoft}
% %
%
% \N{2} strategies are distinguished to compute the \gls{ccf} for a given focal variant.
% The first is referred to as the \emph{hard~breaks} approach, in which it is assumed that the detected breakpoints of an IBD segment are exact.
% In contrast, the second is referred to as the \emph{soft breaks} approach, which considers uncertainty in the inference of breakpoints.
% Details are given below.
%
% \paragraph{Hard breaks.}
%
% The relation derived in \ctref{eq:comgamma} demonstrates that the likelihood function for $\tau$ can be expressed by the Gamma (or Erlang) distribution.
% This was done for each of the clock models, given the respective parameterisation.
%
%
% \paragraph{Soft breaks.}
% The separate consideration of each site enabled the inclusion of additional information.
% A probability value was therefore computed at each of the sites in between the identified breakpoint and the focal variant, on each side.
% Recall that a given IBD segment is defined by a breakpoint interval delimiting the region in which at least \n{1} recombination event was inferred, on each side to the focal variant, but where recombination could have occurred at any site in the sequence.
% For the \gls{fgt} and the \gls{dgt}, this probability was calculated based on the distribution of heterozygosity and homozygosity along the chromosome; thus reflecting a pseudo-probability value.
% In contrast, the \gls{hmm}-based method enabled direct calculation of the posterior probability for the hidden states along the sequence.




\begin{equation}
	\mathcal{L}(\tau \mid \psi,D,\mathrm{I}_L,\mathrm{I}_R)
	~\propto~ \tau^{\mathrm{I}_L + \mathrm{I}_R} \, \euler{-2 \psi D \tau}
\end{equation}
where $\mathrm{I}_L$ and $\mathrm{I}_R$ are indicator functions for the detected breakpoints to the left and right-hand side relative to the position of the focal variant, respectively.
Recall that a given IBD segment may cover the whole distance from the focal site to the end of the chromosome on either side, if no evidence of recombination was found, which is referred to as a \emph{boundary case}.
The indicator functions are therefore defined as
\begin{equation*}
	\mathrm{I}_L =
	\begin{cases}
    ~ 0 & \text{\small if boundary case on left} \\
    ~ 1 & \text{\small otherwise}
  \end{cases}
	\quad \text{and} \quad
	\mathrm{I}_R =
	\begin{cases}
    ~ 0 & \text{\small if boundary case on right} \\
    ~ 1 & \text{\small otherwise}
  \end{cases}\ .
\end{equation*}



Note that the following formulations assume a uniform rate of recombination for simplicity, but where it is straightforward to include a variable recombination rate over the segment to obtain the genetic distance; \eg if a genetic map is available from which genetic lengths can be inferred.


For the formulation of the likelihood function for $\tau$, it is relevant to consider the detected breakpoint as only \n{1} observation in a series of independent observations along the sequence between the focal variant and the inferred breakpoint site.
Since the latter delimits the interval in which at least \n{1} recombination event could have occurred, each variant site in between constitutes a possible recombination point.
For this purpose, let the position of variant sites be denoted by $b_j$, where $j$ is used as an index for sites distal to the focal variant.

Because the probability of finding the first recombination event within the sequence interval is dependent on the probability of recombination at each site, the joint \gls{pdf} is equal to the sum of individual \glspl{pdf}.
Thus, the likelihood for $\tau$ can be written as
\begin{equation}\label{eq:reclike}
	\mathcal{L}(\tau \mid \psi,b_1,b_2,\ldots,b_n)
	~=~ \sum_{j = 0}^{n} 2 \psi \tau \, \euler{-2 \psi \tau (b_j - b_{j-1})}
\end{equation}
where $n$ is the number of variant sites between the focal site at ${j=0}$ and the breakpoint (inclusive).
Note that the factor of 2 in \cref{eq:reclike} comes from the \n{2} lineages considered, where the expected distance to a breakpoint is dependent on the rate of recombination in both lineages.
Since there are \n{2} breakpoints to each side of the focal variant, \cref{eq:reclike} can be used to cover the whole IBD segment by including the sequence intervals to the left and right-hand side independently.
Nonetheless, the likelihood function can be simplified under the assumption that recombination events occurred at (or near) the inferred breakpoints on both sides of the focal variant.


The conjugate relation between \ClockM and \ClockR becomes relevant in \cpref{sec:hardsoft}.





The vast majority of polymorphic sites in the human genome is low in frequency, where a particular allele may only be shared by a small fraction of the population \citep{GenomesProjectConsortium:2012co, 10002015global}.




Briefly, I developed a non-probabilistic algorithm to infer recombination events between pairs of individuals to delimit the interval
\emph{variant-centric}

to learn more about (recent) evolutionary processes and the history of a population.

implication of disease-related variants
The relationship between allele age and allele frequency is of particular interest to the understanding of certain disease traits; for example, if an allele is



the age of an allele is of considerable interest to both population and medical gentics, for example: the time since a disease-related variant


The current age of genetic research is characterised by rapid technological advances that allow the generation of
advances in genomic technologies and the exponentially growing amount of molecular data.

 but proportionally increased requirements
both the generation of data and the growing understanding of the genetic architecture underlying disease traits.
However, the role of rare or low-frequency variants in the aetiology of disease is still less well understood.

The vast majority of polymorphic sites in the human genome is low in frequency, where a particular allele may only be shared by a small fraction of the population \citep{GenomesProjectConsortium:2012co, 10002015global}; however, it has been argued that the importance of rare variants may be inversely proportional to their frequency \citep{Pritchard:2001hw,Manolio:2009jp}.
While it may not be expected that rare variants exhibit large effect sizes, in regards to non-Mendelian disorders, there is some potential that they provide a powerful source of information for direct application in population and medical genetics.

In this chapter, a novel method is developed for the estimation of the evolutionary age of rare or low-frequency variants, for which the previously developed method for inference of IBD segments was crucial; see \cref{ch:sharedhap,ch:generr}.
Notably, no prior knowledge about the demography of the studied population is required.
As in previous chapters, the method prioritises a variant-centric approach, such that individual alleles can be assessed.

The method is first described in detail.
This is followed by an evaluation using simulated data, which allowed to determine the optimal settings.
Then, the method was applied to real data; in particular, data from the \gls{ukb} was available which allowed to investigate the estimated age of reported pathogenic variants in regards to functional consequences.


% But, to date, there has been little attempt to investigate the effects of rare variants on fitness.


In this chapter, I propose a novel statistical method for the estimation of allele age which capitalises on the targeted IBD detection methodology presented in Chapters~3 and~4.
Notably, the method
no prior knowledge about the demography of the studied population is required.
